\documentclass[a4paper,10pt,oneside]{article}
\setlength{\columnsep}{15pt}    %兩欄模式的間距
\setlength{\columnseprule}{0pt}

\usepackage[landscape]{geometry}
\usepackage{amsthm}								%定義,例題
\usepackage{amssymb}
\usepackage{fontspec}								%設定字體
\usepackage{color}
\usepackage[x11names]{xcolor}
\usepackage{xeCJK}								%xeCJK
\usepackage{listings}								%顯示code用的
%\usepackage[Glenn]{fncychap}						%排版,頁面模板
\usepackage{fancyhdr}								%設定頁首頁尾
\usepackage{graphicx}								%Graphic
\usepackage{enumerate}
\usepackage{titlesec}
\usepackage{amsmath}
\usepackage{pdfpages}
\usepackage{multicol}
\usepackage{fancyhdr}
%\usepackage[T1]{fontenc}
\usepackage{amsmath, courier, listings, fancyhdr, graphicx}

%\topmargin=0pt
%\headsep=5pt
\textheight=530pt
%\footskip=0pt
\voffset=-20pt
\textwidth=800pt
%\marginparsep=0pt
%\marginparwidth=0pt
%\marginparpush=0pt
%\oddsidemargin=0pt
%\evensidemargin=0pt
\hoffset=-100pt

%\setmainfont{Consolas}				%主要字型
\setCJKmainfont{PingFangTC-Regular}			%中文字型
%\setmainfont{Linux Libertine G}
\setmonofont{Consolas}
%\setmainfont{sourcecodepro}
\XeTeXlinebreaklocale "zh"						%中文自動換行
\XeTeXlinebreakskip = 0pt plus 1pt				%設定段落之間的距離
\setcounter{secnumdepth}{3}						%目錄顯示第三層

\makeatletter
\lst@CCPutMacro\lst@ProcessOther {"2D}{\lst@ttfamily{-{}}{-{}}}
\@empty\z@\@empty
\makeatother
\lstset{											% Code顯示
language=C++,										% the language of the code
basicstyle=\scriptsize\ttfamily, 						% the size of the fonts that are used for the code
numbers=left,										% where to put the line-numbers
numberstyle=\tiny,						% the size of the fonts that are used for the line-numbers
stepnumber=1,										% the step between two line-numbers. If it's 1, each line  will be numbered
numbersep=5pt,										% how far the line-numbers are from the code
backgroundcolor=\color{white},					% choose the background color. You must add \usepackage{color}
showspaces=false,									% show spaces adding particular underscores
showstringspaces=false,							% underline spaces within strings
showtabs=false,									% show tabs within strings adding particular underscores
frame=false,											% adds a frame around the code
tabsize=2,											% sets default tabsize to 2 spaces
captionpos=b,										% sets the caption-position to bottom
breaklines=true,									% sets automatic line breaking
breakatwhitespace=false,							% sets if automatic breaks should only happen at whitespace
escapeinside={\%*}{*)},							% if you want to add a comment within your code
morekeywords={*},									% if you want to add more keywords to the set
keywordstyle=\bfseries\color{Blue1},
commentstyle=\itshape\color{Red4},
stringstyle=\itshape\color{Green4},
}


\newcommand{\includecpp}[2]{
  \subsection{#1}
    \lstinputlisting{#2}
}

\newcommand{\includetex}[2]{
  \subsection{#1}
    \input{#2}
}


\begin{document}
  \begin{multicols}{3}
  \pagestyle{fancy}

  \fancyfoot{}
  \fancyhead[L]{NCTU-Pusheen}
  \fancyhead[R]{\thepage}

  \renewcommand{\headrulewidth}{0.4pt}
  \renewcommand{\contentsname}{Contents}


  \scriptsize
  \section{Surroudings}
  \includecpp{bashrc}{./Surroudings/bashrc.cpp}

\section{Data\_Structure}
  \includecpp{Segment\ Tree}{./Data_Structure/RangeUpdateSegmentTree.cpp}
  \includecpp{MaxSum\ Segment\ Tree}{./Data_Structure/MaxSumSegmentTree.cpp}
  \includecpp{Persistent\ Segment\ Tree}{./Data_Structure/PersistentSegmentTree.cpp}
  \includecpp{Treap}{./Data_Structure/Treap.cpp}
  \includecpp{Sparse\ Table}{./Data_Structure/SparseTable.cpp}
  \includecpp{BIT}{./Data_Structure/FenwickTree.cpp}
  \includecpp{BIT\_2D}{./Data_Structure/FenwickTree2D.cpp}
  \includecpp{Dynamic\_KD\_tree}{./Data_Structure/Dynamic_KD_tree.cpp}
  \includecpp{Heavy\ Light}{./Data_Structure/HeavyLight.cpp}
  \includecpp{Link\_Cut\_Tree}{./Data_Structure/Link_Cut_Tree.cpp}

\section{Graph}
  \includecpp{LCA}{./Graph/LCA.cpp}
  \includecpp{BCC\_edge}{./Graph/BCC_edge.cpp}
  \includecpp{Tarjan}{./Graph/Tarjan.cpp}
  \includecpp{Two\_SAT}{./Graph/Two_SAT.cpp}
  \includecpp{Min\ Mean\ Cycle}{./Graph/MinMeanCycle.cpp}
  \includecpp{Mahattan\ MST}{./Graph/MahattanMST.cpp}

\section{Flow\_Matching}
  \includecpp{KM}{./Flow_Matching/KM.cpp}
  \includecpp{Min\_Cost\_Max\_Flow}{./Flow_Matching/Min_Cost_Max_Flow.cpp}
  \includecpp{Ford\_Fulkerson}{./Flow_Matching/Ford_Fulkerson.cpp}
  \includecpp{Hungarian}{./Flow_Matching/Hungarian.cpp}
  \includecpp{Hopcroft\_Karp}{./Flow_Matching/Hopcroft_Karp.cpp}
  \includecpp{SW\_MinCut}{./Flow_Matching/SW_MinCut.cpp}
  \includecpp{Dinic}{./Flow_Matching/Dinic.cpp}
  
\section{Geometry}
  \includecpp{Geometry}{./Geometry/Geometry.cpp}
  \includecpp{Hyperbola\_Geometry}{./Geometry/HyperbolaGeometry.cpp}
  \includecpp{Closest\_Pair}{./Geometry/ClosestPair.cpp}
  \includecpp{Smallest\_Circle}{./Geometry/SmallestCircle.cpp}
  \includecpp{旋轉卡尺}{./Geometry/旋轉卡尺.cpp}
  \includecpp{MinRect}{./Geometry/MinRect.cpp}
  \includecpp{Rectangle\_Union\_Area}{./Geometry/Rectangle_Union_Area.cpp}


\section{Math}
  \includecpp{ModInv}{./Math/ModInv.cpp}
  \includecpp{EulerFunction}{./Math/EulerFunction.cpp}
  \includecpp{ax+by=gcd(a,b)}{./Math/ax+by=gcd(a,b).cpp}
  \includecpp{MillerRabin}{./Math/MillerRabin.cpp}
  \includecpp{質因數分解}{./Math/質因數分解.cpp}
  \includecpp{PrimeList}{./Math/PrimeList.cpp}
  \includecpp{Matrix}{./Math/Matrix.cpp}
  \includecpp{Simplex}{./Math/Simplex.cpp}
  \includecpp{Expression}{./Math/Expression.cpp}  
  \includecpp{FFT}{./Math/FFT.cpp}
  \includecpp{NTT}{./Math/NTT.cpp}
  \includecpp{Find\_Real\_Root}{./Math/FindRealRoot.cpp}
  \includecpp{Karatsuba}{./Math/Karatsuba.cpp}
  \includecpp{Discrete\_sqrt}{./Math/Discrete_sqrt.cpp}
  \includecpp{SG}{./Math/SG.cpp}
  \includecpp{外星模運算}{./Math/外星模運算.cpp}


\section{String}
  \includecpp{RollHash}{./String/RollHash.cpp}
  \includecpp{Trie}{./String/Trie.cpp}
  \includecpp{Z}{./String/Z.cpp}
  \includecpp{KMP}{./String/Kmp.cpp}
  \includecpp{AC自動機}{./String/AC自動機.cpp}
  \includecpp{Suffix\_Array}{./String/suffix_array.cpp}
  \includecpp{BWT}{./String/BWT.cpp}
  \includecpp{Manacher}{./String/Manacher.cpp}
  \includecpp{LPS}{./String/LPS.cpp}
  
\section{DP}
  \includecpp{DP\_1D1D}{./DP/DP_1D1D.cpp}
  \includecpp{Bounded\_Knapsack}{./DP/Bounded_Knapsack.cpp}
  \includecpp{LCIS}{./DP/LCIS.cpp}  

\section{Other}
  \includetex{Reminder}{./Other/Reminder.tex}
  \includecpp{莫隊算法\_區間眾數}{./Other/莫隊算法_區間眾數.cpp}
  \includecpp{BuiltIn}{./Other/BuiltIn.cpp}

  \clearpage
  \end{multicols}
  \newpage
  \begin{multicols}{3}
  \enlargethispage*{\baselineskip}
  \begin{center}
    \Huge\textsc{NCTU-Pusheen Codebook}
    \vspace{0.35cm}
  \end{center}
  \tableofcontents
  \end{multicols}
  \clearpage
\end{document}
